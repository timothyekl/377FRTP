\documentclass{article}

\usepackage{amsmath}
\usepackage{amsfonts}
\usepackage{array}
\usepackage{mdwlist}
\usepackage{wasysym}
\usepackage{fancyhdr}
\usepackage{graphicx}

\usepackage{fancyvrb}
\usepackage{cite}
\usepackage{color}
\usepackage[colorlinks,backref]{hyperref}
\usepackage{minted}

\pagestyle{fancy}
\headheight 35pt
\begin{document}

\lhead{\textbf{CSSE377}}
\rhead{Functional Refactoring \\ November 2010}

\vspace*{60mm}
\begin{center}
{ \huge \textbf{Refactoring Techniques \\ in Functional Languages} } \\
{ \large \textit{Pete Brousalis, Tim Ekl, Tom Most, David Pick, Eric Stokes} }
\end{center}

\newpage{}

\tableofcontents

\newpage{}

\section{Introduction}

Refactoring, the process of rewriting program code, is a generally accepted practice in the programming world. Programmers will refactor for a variety of reasons: to improve code readability, to change program structure, to increase maintainability, and to reduce complexity, among others. However, the majority of existing refactoring techniques, including those put forth by the ``Gang of Four'', focus largely on object-oriented code; furthermore, university curricula tend to focus on those refactoring practices, given the prevalence of object-orientation as a programming paradigm.

Of late, however, functional languages have begun to emerge as strong competitors to their object-oriented peers in a variety of environments. Many large-scale websites such as Facebook and Twitter use functional programming and paradigms in their application stacks; MapReduce, the core of Google's search technology and scalability, has its root in functional technologies; and even object-oriented languages such as Python and Ruby have begun to adopt several functional practices. As the functional paradigm becomes more prominent, coders must be able to apply refactoring techniques to code where the Gang of Four patterns may no longer apply.

This paper provides a brief insight into the state of refactoring in five different functional languages: Clojure, Erlang, Haskell, Scala, and Scheme. The authors review each language and its applications, then provide examples of common refactoring techniques in use with each of these languages, as well as how those techniques relate to the practice of functional programming as a whole.

\section{History}

The concept of functional languages began with the development of lambda calculus, a mathematical concept used to encapsulate certain ideas about computation. The vast majority of functional languages today are based on ideas derived from lambda calculus \cite{combinatorylogic}.

The first true functional language was Lisp, developed in the late 1950s for IBM scientific computers at MIT. Lisp introduced many concepts common to functional languages now, and spawned a number of dialects, including Common Lisp and Scheme \cite{historyoflisp}. Over the next few decades, a small number of prevalent functional languages were introduced, including APL and ML. The early 1990s saw a large increase in the number of functional languages available with the advent of J and K, as well as the beginning of the standardization process for Haskell.

Today, functional languages remain somewhat of a specialty: of the top 20 programming languages in use today, the only functional language to hold a spot is Lisp at \#13, with just over 1\% of existing worldwide production code. However, Scheme, Haskell, ML, APL, and Scala all appear in the top 50 programming languages, and most of these are gaining in popularity \cite{TIOBE}.

\documentclass{article}

\usepackage{amsmath}
\usepackage{amsfonts}
\usepackage{array}
\usepackage{mdwlist}
\usepackage{wasysym}
\usepackage{fancyhdr}
\usepackage{graphicx}

\pagestyle{fancy}
\headheight 35pt
\begin{document}

\lhead{\textbf{CSSE377 \\ Functional Refactoring}}
\rhead{November 2011 \\ Clojure}

\section{Clojure}

Clojure is a dialect of Lisp, developed as a functional companion to Java that targets existing Java Virtual Machines (JVMs) and is therefore compatible with preexisting Java projects. 

\subsection{Language Background}

Clojure first appeared in 2007 as a personal project of Rich Hickey, who explained that he wanted a language which:

\begin{itemize*}
\item was a Lisp dialect
\item supported functional programming paradigms
\item was ``symbiotic'' with an existing, established platform
\item was designed for concurrency
\end{itemize*}

No such language existed to Hickey's satisfaction, and thus Clojure was born. Since inception, Clojure has rapidly expanded to become a complete Lisp dialect, unconstrained by what Hickey calls the ``slow innovation''\cite{1} of both Common Lisp and Scheme since their respective standardizations. Since Clojure is not restricted by the scope of those dialects' standards, it can become a more flexible language, so long as it adheres to the JVM specification in compilation.

Clojure is a compiled language. Once written, Clojure code is converted to JVM bytecode that can be natively interpreted by any JVM, including the Sun standard virtual machines. As compiled, it supports the complete Java specification, including type hinting and inference; it avoids reflection where possible.

Furthermore, as a Lisp dialect, Clojure is highly functional. It contains a lambda calculus core, and extends the paradigm of ``code as data'' to Java maps and vectors. All data types implemented in Clojure are immutable and persistent, making both recursion and concurrency easy and reliable.

Finally, Clojure adheres to Java's polymorphic behavior, providing the ability to subclass and abstract certain parts of written code. It does not provide its own class system, instead preferring to provide many methods that operate on a relatively small number of classes and objects. Hickey asserts that ``inheritance is \textit{not} the only way to do polymorphism''\cite{1}, and attempts to get away from the object-oriented paradigm while still embracing the underlying Java platform.

\subsection{Refactoring in Clojure}

As with most functional languages, Clojure is not subject to the usual set of refactoring techniques available for object-oriented architectures. Instead, Clojure has developed its own set of potential refactorings, nurtured and encouraged by Hickey.

\subsubsection{Rename refactoring}

One of the simplest refactoring techniques available in Clojure is a simple rename. While relatively trivial to do, this is also a fairly powerful technique - Hickey has gone so far as to call it ``the number one refactoring''\cite{2}. The simplicity of the technique comes from the static typing of names within any program. For example, when attempting to refactor a variable \verb!bar! within the \verb!foo! namespace (written as \verb!foo/bar!), it is trivial for a good IDE or compiler to determine what instances of \verb!bar! are syntactically equivalent to the target \verb!bar! by checking the variables' scope and namespaces. Hence, for the Clojure snippet:

\begin{verbatim}
(def bar 1)

(defn foo[]
    (let [bar 2]
        (println bar)))
\end{verbatim}

Renaming the inner variable \verb!bar! would resolve to the local instance defined within the \verb!let! statement (and hence apply the rename refactoring to that \verb!let!), while renaming the instance of \verb!bar! defined in the \verb!def! statement would have no other effects (since there are no inner instances of \verb!bar!) that resolve to that definition. The results of these two refactorings are shown as follows:

\begin{tabular}{m{2in} m{2in}} \\
\begin{verbatim}
// Rename def'd bar
(def baz 1)

(defn foo[]
    (let [bar 2]
        (println bar)))
\end{verbatim}
&
\begin{verbatim}
// Rename let'd bar
(def bar 1)

(defn foo[]
    (let [baz 2]
        (println baz)))
\end{verbatim}
\end{tabular}

The advantage of this refactoring technique lies primarily in its potential to enhance code readability and maintainability - a developer who can make sense of variable naming schemes and understanding the meaning underlying a piece of code is much more likely to be productive and make workable changes than one who has no idea what's going on in a code snippet. Furthermore, this technique is especially powerful because of the simplicity of its application - a sufficiently ``intelligent'' editing environment can easily handle this sort of change, and in fact several projects exist on GitHub that handle precisely this refactoring (most notably \verb!clojure-refactoring! by Tom Crayford \cite{3}).

\subsection{Conclusion}

Wrap up your section with a 0.5--1 page conclusion that recaps refactoring in your language and discusses any emerging concepts or practices in your language.

\begin{thebibliography}{9}
\bibitem{1} Hickey, Richard. "Clojure: Rationale." Clojure - Rationale. Web. 28 Oct. 2010. \verb!<http://clojure.org/rationale>!.
\bibitem{2} Hickey, Richard. "Re: Refactoring? - Richard Hickey - com.googlegroups.clojure - MarkMail." Mailing List. com.googlegroups.clojure. Google Groups, 23 July 2008. Web. 31 Oct. 2010. \verb!<http://markmail.org/message/btbnpxvuyob5xrig>!.
\bibitem{3} Crayford, Tom. tcrayford's Clojure-refactoring at Master - GitHub. Clojure-refactoring. 23 Jan. 2010. Web. 31 Oct. 2010. \verb!<http://github.com/tcrayford/clojure-refactoring>!.
\end{thebibliography}

\end{document}


\documentclass{article}
 
\usepackage{amsmath}
\usepackage{amsfonts}
\usepackage{array}
\usepackage{mdwlist}
\usepackage{wasysym}
\usepackage{fancyhdr}
\usepackage{graphicx}
 
\pagestyle{fancy}
\headheight 35pt
\begin{document}
 
\lhead{\textbf{CSSE377 \\ Functional Refactoring}}
\rhead{November 2011 \\ Erlang}
 
\subsection*{Language Background}
 
Erlang has been around since around 1987 and was considered a declarative language at that time.\cite{AR07}  A declarative language that is being described here is a high level language that prefers to state what is to be done rather than how to do it.\cite{EB10} Declarative languages are considered to be a higher genre of both logic and functional languages. As time progressed, Erlang was considered a functional language with concurrency, as of 1995.\cite{AR07} This simply means that the language was further defined, creating a more focused view and better goals of the language. Functional languages differ from declarative, in the style of the language. Functional programming languages have deep roots in mathematics, and thus the style is very mathematical.\cite{EB10} These languages are comprised largely of functions and applying arguments to them. Later, as of 2005, Erlang is now considered a concurrent language whose components are written in a functional style.\cite{AR07} The shift in titles means that as of recent, people are considering the concurrent style of Erlang to be the primary focus of the language.  There are many other functional languages, but not as many that focus primarily on concurrency.  Concurrency makes debugging, analyzing and testing programs espically difficult which is why it is so important to refactor the code.  With many concurrent processes going on at the same time, debugging is near useless on a high level, which is why being able to break our code apart and make it more readable is so important.  With concurrent languages, it is also important to keep code efficiency and reusability high, which refactoring can help a bunch with also.
 
There have been many other milestones that have brought Erlang to where it is today. The first true release of Erlang to users took place in 1991.  This was after almost 10 years of development, once Erlang was fast enough to be considered usable.\cite{ERHIST} Only one year later, Erlang was ported to VxWorks, PC and Mac.\cite{ERHIST} This allowed many more users to take a look and expierement with Erlang. The increase of platforms was a huge point for Erlang, with it now being available on home computing platforms, anyone with any one of these computers could try it out.  Previously, Erlang had been restricted to users in educational vocations, which is a great place for a test bed, but not for a growing language. In an additional year, 1993, distribution was added.\cite{ERHIST} This allowed users to run a uniform distribution of Erlang on a non-uniform computing cluster.  Effectively, this would allow us to run a single program on many computers thus infinitely expanding the use of Erlang. Now it was not a toy, it was a real tool. The primary use of Erlang was now to be able to perform very intense concurrent distributed algorithm analysis.
 
The primary use of Erlang was aimed at telephony applications, as it's initial developer was Ericsson (one of Sweden's telecomm companies).\cite(ERWIK) The name Erlang has actually been attributed to both the mathematician Agner Krarup Erlang and the abbreviation of the company who designed the language ERicsson along with LANGuage at the end. With a commercial background, Erlang was destined to succeed, however just because something had funding, did not guarantee success. Erlang, being designed for telecommunications use, made sure to provide low level, cheap threading.  Processes comunicate solely by message-passing rather than using shared variables. Due to this nifty way of handling communications throughout multiple portions of code, there is no need for locking. Locks allow certain pieces of code to be locked so nothing else can touch them, and are required for most concurrency implementing languages.\cite{LOWIK} This way, we can maintain thread safety while still creating an environment that allows refactoring to be done. It is important to note that while Erlang does focus a lot on threads and distribution, it is still considered a high level language, as it still supports almost every aspect of functional programming. 
 
Erlang provides eight primitive data structures that we can use, the two most uncommon being: Atoms and Ports. The other types, such as Integers, Floats, References and the like are very similar to their C equivalents. There are also two compound datatypes: Tuples and Lists.  These simple language attributes show that Erlang, as a whole, is a very lightweight language, and is not intended for everyday use.  On the other hand, this light weight of the language truly lends itself to being able to be run across many platforms simultaneously and definitely makes running multiple threads hyper efficient. The Erlang language as a whole is very compact, but code written in it isn't always, this is why we need to refactor.
 
\subsection*{Refactoring in Erlang}
 
The refactoring process for Erlang is actually much more defined than one would expect, including an entire application being written solely for the purpose of refactoring Erlang. This piece of software supports 24 different refactoring steps which each have an important function.
 
Erlang's syntax makes it espically difficult to refactor, which is why tools provide a great way to do it for you in the correct programming environments.  Let's take a look at some very basic refactoring examples and why the Erlang syntax makes it difficult.
 
\begin{verbatim}
myFunction(false) ->
 doCoolStuff(),
 doMoreCoolStuff();
myFunction(true) ->
 doCoolStuff().
\end{verbatim}
 
Let's say we simply wanted to move one of these pieces of code from one branch to another, in a language like Java, we'd siply grab that line and move it.  Not in Erlang.  Due to the nature that effectively a function's syntax propegates throuout the entirety of the function, we have to do this:
 
\begin{verbatim}
myFunction(false) ->
 doCoolStuff();
myFunction(true) ->
 doMoreCoolStuff(),
 doCoolStuff().
\end{verbatim}
 
Make note the ordering of the semocolon, comma and period.  This even makes it difficult if we simply want to change the order in which the two functions are called!
 
\begin{verbatim}
myFunction(false) ->
 doCoolStuff();
myFunction(true) ->
 doCoolStuff(),
 doMoreCoolStuff().
\end{verbatim}
 
Now, one further operation on this bit of code; change the order in which the operation that happens if true is passed or if false is: go.
 
\begin{verbatim}
 
myFunction(true) ->
 doCoolStuff(),
 doMoreCoolStuff();
myFunction(false) ->
 doCoolStuff().
\end{verbatim}
 
Wow, there it is, somehow, Erlang made such simple refactoring operations quite complex! Now let's do some refactoring with working code; we can start with a simple program that prints the items in a list.  This is a simple task to perform in any language, but can be a bit daunting in a functional language.
 
\begin{verbatim}
-module (example).
-export([pl/1]).
pl([H|T]) ->
   io:format("�p\n", [H]),
   pl(T);
pl([]) -> true.
\end{verbatim}
 
This function, printTheList takes two parameters, H and T and prints them to the console.  H is the string variable format and T is the remainder of the list.  This function recursively calls itself.  In this refactoring example, one of the simplest refactoring techniques allows us to make this function much more readable by any person.  We will simply rename the H, T and function name to output, remainingList and printTheList, respectively.
 
\begin{verbatim}
-module (example).
-export([printTheList/1]).
printTheList([output|remainingList]) ->
   io:format("�p\n", [output]),
   printTheList(remainingList);
printTheList([]) -> true.
\end{verbatim}
 
Now, the second anyone looks at this code, it is instantly obvious what it does, regardless of their language background!
   
Basic refactoring techniques can be applied to any code in any language, and as Erlang is a high level functional language, many of the techniques that can be applied in Object Oriented programming, can be applied here as well!
 
\begin{verbatim}
get_submatrices(Puzzle) -> 
 NSquared = length(Puzzle),
 N = trunc(math:sqrt(NSquared)),
 BinaryEncoded = 
   [split_into_ns(list_to_binary(lists:map(fun remove_unk/1, Row)), N, []) || Row <- Puzzle],
 BinaryTrans = transpose(BinaryEncoded),
 [lists:map(fun add_unk/1, binary_to_list(BinBlock))
   || BlockCol <- BinaryTrans, 
   BinBlock <- split_into_ns(list_to_binary(BlockCol), NSquared, [])].
\end{verbatim}
 
One of the big selling points of Erlang is its efficiency. It can be very fast. This comes with the same warning that C can be very fast, if you're not wasting memory or resources. One of the biggest places we loose memory when we're hyper-optimizing our code is in required variable assignments. This is the point where we have to step back and take a look at what we're really trying to refactor.  We definitely want more testible code, we definetly want fast code, and we also want readable code.  In this instance when we're valuing testability and speed over readability, in many cases these extra variable assignments are just extra fluff that we really don't need.  We can see there is one location above where we assign NSquared that I kept in the bottom block. This is again, for speed reasons.  Although it costs us memory to store this value, we're only computing it once, rather than twice in this small method snippit.
   
\begin{verbatim}
get_submatrices(Puzzle) -> 
 NSquared = length(Puzzle),
 BinaryEncoded = 
   [split_into_ns(list_to_binary(lists:map(fun remove_unk/1, Row)),
              trunc(math:sqrt(NSquared)), []) 
       || Row <- Puzzle],
 BinaryTrans = transpose(BinaryEncoded),
 [lists:map(fun add_unk/1, binary_to_list(BinBlock))
   || BlockCol <- BinaryTrans, 
   BinBlock <- split_into_ns(list_to_binary(BlockCol), NSquared, [])].
\end{verbatim}
    
 
\subsection*{Conclusion}
 
Refactoring in Erlang has definetly proved to be a relatively difficult, but important beast.  When taking a look at when it is required to refactor, we need to look at what systems need to be refactored.  At the simplest level; every system can use some form of refactoring. This is because the art of refactoring can have different meanings in different contexts.  Refactoring can make code faster, smaller, more readable, more testable or even easier to package. One of the biggest problems or difficulties that comes with refactoring is the simple fact that many of these types of refactoring cannot take place with others being used.  It can be problematic to try and make code smaller, when we're making variable names longer.  Erlang suffers from these problems among the many others, primarily with it's syntax.  Erlang is very difficult to refactor due to the nature of how functions are written. It is also very important to refactor Erlang; a language used primarily for distributed concurrent telecommunications systems. No one ever wanted to have only one person talking on a telephone, which is why unrefactored Erlang code can result in one of the largest debugging nightmares on the planet. When we have the multiple variations of our code running over a distributed system, and a specific server cluster goes down, how do we know what went wrong? Well it is very certain that large chunks of code will surely lead to chaos. This chaos is not only that of debugging or multiple process, but especially if we had to deal with locks. If an alternative language had been chosen, we would have to deal with locking and unlocking resources on an enormous scale, hundreds of thousands of calls, hundreds of servers, billions of bytes of memory. Erlang was a language designed with distributed concurrency in mind, which is absolutely perfect for the telecommunications world.
 
\begin{thebibliography}{9}
\bibitem{AR07} Armstrong, Joe. A History of Erlang. Tech. Print.
\bibitem{EB10} ``computer programming language.'' Encyclopaedia Britannica. 2010. Encyclopaedia Britannica Online. 31 Oct. 2010 \verb!<http://www.britannica.com/EBchecked/topic/130670/computer-programming-language>!.
\bibitem{ERHIST}``History of Erlang.'' Erlang Programming Language, Official Website. Web. 31 Oct. 2010. \verb!<http://www.erlang.org/course/history.html>!.
\bibitem{ERWIK} ``Ericsson.'' Wikipedia, the Free Encyclopedia. Web. 31 Oct. 2010. \verb!<http://en.wikipedia.org/wiki/Ericsson>!.
\bibitem{ERLWIK} ``Erlang (programming Language).'' Wikipedia, the Free Encyclopedia. Web. 31 Oct. 2010. \verb!<http://en.wikipedia.org/wiki/Erlang_(programming_language)>!.
\bibitem{LOWIK} ``Lock (computer Science).'' Wikipedia, the Free Encyclopedia. Web. 31 Oct. 2010. \verb!<http://en.wikipedia.org/wiki/Lock_(computer_science)>!.
\end{thebibliography}
 
\end{document}
?

\section{Haskell}

Haskell is a purely functional language known for its laziness and powerful static type system that has served as an academic testbed for twenty years.

\subsection{Language Background}

The Haskell has its roots in academia, having been created in 1987 to replace several similar non-strict functional languages to reduce duplication of effort and provide a common baseline\cite{hudak2007history}.
It is named after the mathematician Haskell Curry, who also lent his name to the \textit{currying} so fundamental to the language.
This academic background, along with an informal community determination to	``avoid success at all costs,'' meaning to avoid widespread commercial use so as to allow the language to continue evolution, has kept Haskell out of common use in industry today, it has seem some use\cite{haskellinindustry}.

Aside from being a functional language, one of the most unique features of Haskell is \textit{laziness}, more formally known as \textit{non-strict evaluation}.
This means that expressions are not evaluated until their values are required, and is possible because functional purity means that a function's output depends only on its input, and not on any external context that may change depending on when it is executed.
This deferral of execution is both an optimization which sacrifices memory for processor time and enables new programming techniques.
One example is that of infinite sequences.
Though this concept is present in other languages --- Python's \verb!itertools.count! being one example of language tools that create explicit laziness via the iterator concept\cite{pythonitertools} --- it does not require special syntax or the contortions common to programming with iterators.
The most trivial example of this is a sequence of increasing integers, which can be generated in Haskell as follows:

\begin{verbatim}
seq n = n : seq(n + 1)
\end{verbatim}

\noindent The \verb!:! operator is like Scheme's \verb!cons!; it creates a new linked-list node.
To generate a sequence starting at zero, call \verb!seq 0!.
\verb!take 5 (seq 0)! would evaluate to the list \verb![0, 1, 2, 3, 4]!.

A more interesting example of this is a pseudo-random number generator.
Instead of implementing a random number generator as an object that encapsulates its state, the generator is simply a function which returns an infinite list containing the numeric sequence based on a given seed value.
Because of lazy evaluation the infinite loop that the function uses to generate the list is only executed as elements in the list are accessed.


It is often said that a Haskell program that compiles --- once it compiles --- is more likely to be correct than in most languages.
This is because Haskell's type system is extremely strict, yet expressive, allowing programs to be phrased in terms of types.
\textit{Typeclasses} are one of the language's main innovations\cite{hudak2007history}.
They provide functionality like a combination of interfaces and abstract classes --- a type can implement more than one typeclass, like an interface, but the typeclass can include implementation like an abstract class.
One concept familiar from object-oriented languages is missing: the type system has no concept of inheritance.
\textit{Algebraic types} are another interesting feature.
They allow specification of several alternatives within a single type:

\begin{verbatim}
-- A binary tree type
data BinTree a =
        ExtNode
      | IntNode a (BinTree a) (BinTree a)
\end{verbatim}

\noindent This snippet defines a new datatype, where \verb!a! is a \textit{type variable}, which is like the type parameter in Java generics or C++ templates.
Pattern matching can be performed in a function definition as follows.

\begin{verbatim}
-- Return a list of the values in the tree inorder
inorder (ExtNode) = []
inorder (IntNode b left right) =
    inorder left ++ [b] ++ inorder right
\end{verbatim}

\noindent The \verb![]! syntax crates a list, and the binary operator \verb!++! concatenates lists.

Haskell has another important form of type declaration: \verb!data!:

\begin{verbatim}
data ThreeStrings = ThreeStrings String String String
    deriving (Show)
\end{verbatim}

\noindent Here, \verb!data ThreeStrings! declares the type name, and the second \verb!ThreeStrings! is the name of the \textit{type constructor}.  The constructor can be used as a function, taking three \verb!String! values.  \verb!deriving (Show)! indicates that this type implements the \verb!Show! typeclass, which gives it an implementation of the \verb!show! function to print it.

One more feature of Haskell bears mentioning: type inference.  Like the modern object-oriented languages D and Go, Haskell does not force users to specify types when they can be inferred automatically.  It does allow type annotations, however, and these are commonly used on functions as both documentation and to ensure correctness:

\begin{verbatim}
inorder :: BinTree a -> [a] -- <- type annotation
inorder (ExtNode) = []
inorder (IntNode b left right) =
    inorder left ++ [b] ++ inorder right
\end{verbatim}

Other distinctive Haskell features include monads, the way that the pure language manages I/O and other side effects, and automatic parallelization/concurrency, though there is not enough space to discuss them here.

\subsection{Refactoring in Haskell}

As is fitting, refactoring Haskell code tends to involve manipulation of function scope or types.  Two such methods are represented below.

The following example module will be used to demonstrate refactoring in Haskell.  It exposes a \verb!BinTree! data type representing a binary tree and several functions to compute traversals of it.  This example is adapted from an assignment given in Curt Clifton's CSSE403 course\cite{clifton2010}.

\begin{verbatim}
module BinTree
    (
      BinTree
    , ExtNode
    , IntNode
    , inorder
    , levelorder
    ) where

data BinTree a =
      ExtNode
    | IntNode a (BinTree a) (BinTree a)

inorder :: BinTree a -> [a]
inorder (ExtNode) = []
inorder (IntNode a left right) =
    inorder left ++ [a] ++ inorder right

levelorder :: BinTree a -> [a]
levelorder (ExtNode) = []
levelorder (IntNode a left right) =
    a : (loHelper [left, right])

loHelper :: [BinTree a] -> [a]
loHelper [] = []
loHelper (ExtNode : queue) = loHelper queue
loHelper ((IntNode a left right): queue) =
    a : loHelper (queue ++ [left, right])
\end{verbatim}

The \verb!module! declaration at the top of the file includes a list of the symbols exported by the module.
Any not listed here are considered private and are inaccessible elsewhere.

\subsubsection{Demoting a Definition}

The most immediate change to be made is to move the definition of \verb!loHelper! to the local scope of the \verb!levelorder! function.  
While this might seem like a contrived example, this operation is common when writing Haskell functions, as maintaining the helper function externally makes it testable in the interactive prompt.
This is called \textit{demoting a definition}\cite{li2006refactoring}, and is also the reverse of  is the reverse of the refactoring known as ``lambda-lifting'' (or $\lambda$-lifting)\cite{haskellwikilifting}.
The mechanics of this refactoring are straightforward; first move the definition to within \verb!levelorder! and then renaming it to improve readability:

\begin{verbatim}
levelorder :: BinTree a -> [a]
levelorder (ExtNode) = []
levelorder (IntNode a left right) =
    a : (helper [left, right])
    where
        helper :: [BinTree a] -> [a]
        helper [] = []
        helper (ExtNode : queue) = helper queue
        helper ((IntNode a left right): queue) =
            a : helper (queue ++ [left, right])
\end{verbatim}

\subsubsection{Concrete to Abstract Data Type}

One potential problem with this code is that the internals of the \verb!BinTree! data type are exposed to the world.
Clients of the module can call the \verb!IntNode! and \verb!ExtNode! constructors and perform pattern matching on the tree structure.
If the tree represented is balanced, one alternative representation would be:

\begin{verbatim}
data BinTree a = 
      LeafNode a
    | IntNode a (BinTree a) (BinTree a)
\end{verbatim}

\noindent But if the maintainer of the module were to convert it to use this representation, all client code would have to be modified to match against the new structure\cite{thompson2005refactoring}.
In order to ease the introduction of such changes, the \verb!BinTree! data type can be made abstract, hiding its implementation details outside the module.
This is a composite refactoring, involving modifying the module's exports and adding new functions that expose semantic information about the data structure.
By removing the raw type constructors from the export list they can be hidden from external modules.
Finally, functions that use pattern matching are rewritten to use the new functions instead of pattern matching on the data structure.

First, modify the module declaration to remove the type constructors and add some of the functions to be written:

\begin{verbatim}
module BinTree
    (
      BinTree
    , isInternal
    , isExternal
    , left
    , right
    , value
    , internalNode
    , externalNode
    , inorder
    , levelorder
    ) where
\end{verbatim}

Second, add the new functions by which the data structure is exposed.  As the actual representation hasn't been changed, the constructors exposed externally can simply alias the internal ones.

\begin{verbatim}
isInternal (IntNode a) = True
isInternal _           = False

isExternal (ExtNode) = True
isExternal _         = False

value      (IntNode v _ _) = v
left       (IntNode _ l _) = l
right      (IntNode _ _ r) = r

-- Simply delegate to the current constructors
internalNode = IntNode
externalNode = ExtNode
\end{verbatim}

Finally, rewrite functions to use the new interface.  This example uses the Haskell syntactic feature known as a \textit{guard}, which functions much like a string of if..else statements in another language.

\begin{verbatim}
inorder :: BinTree a -> [a]
inorder t
    | isExternal t = []
    | otherwise    = inorder (left t) ++ [value t] ++ inorder (right t)
\end{verbatim}

\subsection{Conclusion}

Haskell is unique among the popular functional languages for its laziness and powerful type system, leading to increased interest in the language in recent years after twenty spent in academia.
Refactoring in Haskell is a fairly young practice, as projects large enough to call for a formal approach have only recently emerged.
The only tool available to assist is HaRe, the Haskell Refactorer\cite{li2006refactoring}, a product of research at the University of Kent.
It includes support for the refactorings described above, as well as several others.
The purely functional nature of the language tends to limit the complexity of refactorings, which are typically fairly simple mechanical transformations when dealing with functions, but can be more complex when manipulating types.


\section{Lisp}

\subsection{Language Background}

Originally created in 1958, Lisp is the second oldest high-level programming language still in use today. The name Lisp comes from ``list processing'' as linked-lists are one of the most important data structures in the language, along with the fact that the entire source code for the language is made up of lists. Since List source code is made up of lists, the language allows a developer to manipulate source code as a data structure. This lead to the creation of macro systems which allow programmers to create new syntax, and domain-specific languages based on the original Lisp language\cite{evolutionoflisp}.

A domain-specific language is one that is based on a language such as Lisp, but provides an extended syntax for handling a specific problem domain. For instance, if someone were to modify the original list language with specific functions for creating web applications, they would have created a domain-specific language\cite{domainspecific}.

Lisp and languages based on it can easily be recognized by their extreme use of parenthesis as well as the S-expressions that are used to write in the language. An S-expression or symbolic-expression (commonly referred to as \textit{sexps}) are list-based data structures, used to represent semi-structured data. A simple example of a sexp would be \verb!(+ 2 2)!. This sexp highlights Lisp prefix notation, where unlike most other languages operators come first in the sexp while everything else is treated as data. The combination of prefix-notation and sexps makes Lisp extremely easy to parse and led to the creation of two extremely well known functions: \verb!car! and \verb!cdr!. \verb!car! returns the first part of an sexp, while \verb!cdr! will return the data portion of the sexp.


\subsection{Refactoring in Lisp}

\subsubsection{Extract Function}

One of the easiest programming pitfalls to fall into is the creation of functions that are longer than they need to be or should be. While there are several reasons that a developer should attempt to write shorter methods, the first and most important is that it increases the likelihood that other functions can use a function. This one simple refactoring technique will allow higher level functions to read much better since they will call well named functions which can actually be read, similarly to how a comment would. While this idea does take some getting used to if a developer is used to writing longer functions, the addition of small functions with good, clear names will greatly improve the readability, modifiability, and understandability of a code base. Consider this Lisp snippet:

\begin{verbatim}
(defun list-to-comma-delimited-string (list)
  (let ((output (format nil "~A" (car list))))
    (dolist (elem (cdr list))
      (setq output 
        (concatenate 'string output "," 
          (format nil "~A" elem)))))
\end{verbatim}

This function is a good example of one that is trying to do too much. By calling a second function designed for the formatting of strings, the developer can make the original function much easier to read and understand:

\begin{verbatim}
(defun list-to-comma-delimited-string (list)
  (format nil "~{~A~^,~}" list))
\end{verbatim}

\subsubsection{Rename Function / Variable}

The first refactoring technique introduced in this section was the idea of short functions. However, in order for short functions to work properly as a refactoring technique, all functions and variables must be named in a clear manner that gives the developer an idea of their purpose. Functions need to be named in such a way that communicates their intention. A good way to do this is to think of what a comment would be for a function and try and incorporate that into the name of the function. This is an extremely important technique that can take some time to master. While attempting to understand this technique, a developer should not be happy with their first attempt at naming a function. If it is clear that a function will end up preforming a different function from when it was named, simply change the name. The developer must remember that the code they are writing is meant for humans to read, and the best way to make their code understanding is through well named functions and variables\cite{lisprefactoring}. 

A simple example of this would be the following function:

\begin{verbatim}
(defun rev (L)
  (let ((return-value '()))
    (dolist (e L) (push e return-value))
    return-value))
\end{verbatim}

While after reading the code for this function, it may be apparent that it reverses a list, reading a call to this function from another one would not make that fact obvious. This means that as a developer is reading a function which calls this one, they will be forced to read the code for this function to understand the one they are currently looking at. This will tend to break their train of thought and force them to spend a significant amount of extra time reading code that could have been avoided through a function name that made sense\cite{lisprefactoring}.

\begin{verbatim}
(defun reverse-list (list)
  (let ((return-value '()))
    (dolist (e list) (push e return-value))
    return-value))
\end{verbatim}

By naming the function reverse-list instead of rev, it becomes apparent that this function will reverse the passed in list without having to look at the code for it.

Along the same lines, it is just as important to name variables inside of functions well. By naming a variable in clear manner, a developer can make his application much more readable. In Lisp, it is very common to create a variable to pass to a recursive function to act as an accumulator. While the developer has the choice to name this variable whatever they want it can often be helpful to name it something that makes it obvious to another developer that this variable is the accumulator. Looking at the same reverse list function as before we can see this.

\begin{verbatim}
(defun reverse-list (list)
  (let ((r '()))
    (dolist (e list) (push e r))
    r))
\end{verbatim}

In this function the variable \verb!r! acts as the accumulator for the function. However, it is very difficult to figure this out in the function, and once a developer has determined that \verb!r! is the value to be returned, it can be difficult for them to remember that each time they see the variable. In order to combat this issue, a developer should attempt to name their variables in a clear way:

\begin{verbatim}
(defun reverse-list (list)
  (let ((return-value '()))
    (dolist (e list) (push e return-value))
    return-value))
\end{verbatim}

This refactored function is much easier to understand than the first version -- it is very apparent what the purpose of the \verb!return-value! variable is.

\subsubsection{Code Organization}

One of the simplest yet most powerful refactoring techniques is simply how the code is formatted in a file. In Lisp, since every function called must be made in a set of parentheses, it becomes quite easy to miss where one set of parentheses beings and another ends. In order to avoid this confusion, formatting the code such that new function calls appear on a new line and are indented greatly aids in improving the readability of the code. Consider this Lisp code, line-broken only to fit on the page:

\begin{verbatim}
(fomus :output '(:lilypond :view t) :parts (list (make-part :name 
"Piano" :instr :piano :events (loop repeat 10 for off = (random 
30.0) and dur = (1+ (random 3.0)) collect (make-note :off off :dur 
dur :note (+ 60 (random 25)))))))
\end{verbatim}

While this is in fact valid Lisp code, it is extremely difficult to read and understand. By simply adding in a few line breaks and tabs, we can end up with a function that is much clearer.

\begin{verbatim}
(fomus
 :output '(:lilypond :view t)
 :parts
 (list
  (make-part
   :name "Piano"
   :instr :piano
   :events
   (loop repeat 10
	 for off = (random 30.0)
	 and dur = (1+ (random 3.0))
	 collect (make-note :off off :dur dur :note (+ 60 (random 25)))))))
\end{verbatim}

\subsubsection{Remove Magic Numbers}
 
The final Lisp refactoring technique I’m going to cover is replacing a magic number with a constant. Magic numbers are numbers with a specific value which must be used inside of a program. While at first glance it may seem like a reasonable idea to use a number in your program without assigning it to a variable, they can quickly make the code much less readable. This happens because often when magic numbers are used, they will need to change during the development of the application. If the number is used in multiple locations then it must be changed in each location. If the developer forgets to change it in one place it create extremely confusing scenarios which make debugging the application much more difficult. A simple example of this technique can be seen in the following example.
 
\begin{verbatim}
(defun factorial (n)
  (if (<= n 1)
      1
      (* n (factorial (- n 1)))))
\end{verbatim}
 
Looking at this example we can see the use of two seperate magic numbers. In this case both numbers are 1, making it all the more confusing if only the base case for the problem needed to change. It would be an easy to simply change all the 1’s causing a bug in the function.

\begin{verbatim} 
(defun factorial (n)
 (let [base 1] [subtract-value 1]
  (if (<= n base)
      base
      (* n (factorial (- n subtract-value))))))
\end{verbatim}
 
The refactored code here is much clearer and the magic numbers are defined in the let, then called every where else in the function.
 
\subsubsection{Tests}
 
While these refactoring techniques are very useful when applied correctly, they are only as good as the tests that go along with them. When you refactor code the developer is essentially rewriting there code. In order to be sure that each time a change is made the functionality does not also change, a good suite of tests are required\cite{lisprefactoring}. 
 
Although tests cannot prove that a method works correctly, it would be extremely difficult to write a unit test for every possible input to a function. They can give a reasonable assurance that a function preforms in the intended manor. 
 
In Lisp one of the easiest ways to define tests is through the lisp-unit package. It allows the developer to write tests similar to:
 
\begin{verbatim}
(define-test test-factorial
 (assert-equal 1 (factorial 1))
 (assert-equal 2 (factorial 2))
 (assert-equal 6 (factorial 3))
 (assert-equal 24 (factorial 4)))
\end{verbatim}
 
A suite of tests similar to this one provides the programmer an easy way of ensuring their code works as expected before and after any changes are made. 
 
\subsubsection{Conclusion}
 
While refactoring is important to any language, it is especially important in Lisp. This is true, because of Lisp’s cluttered syntax. Often times when writing code in Lisp, a developer will have difficulty understanding it after working on another function for a period of time. This means that making code more readable, maintainable, and understandable of the utmost importance. In order to ensure the quality of a developers code while writing in Lisp, they must be constantly looking for ways to refactor their code. This will help ensure that future developers have an easier time understanding code that has already been writing, and are better able to incorporate already written functions into their own code.
 
Through these examples we have seen just how much of a difference specific refactorings can make to overall code quality. While these techniques can be applied in many places in all kinds of code, a developer should always be on the lookout for places where they can make their code better. The examples shown in this paper are by no means the only ways code can be refactored, and developers need to be constantly looking for new patterns which can be applied to their code.


\documentclass{article}

\usepackage{amsmath}
\usepackage{amsfonts}
\usepackage{array}
\usepackage{mdwlist}
\usepackage{wasysym}
\usepackage{fancyhdr}
\usepackage{graphicx}

\pagestyle{fancy}
\headheight 35pt
\begin{document}

\lhead{\textbf{CSSE377 \\ Functional Refactoring}}
\rhead{November 2011 \\ Scala}

\section{Scala}

Scala is a multi-paradigm programming language designed to be a ``scalable language,'' meaning that it grows with the demand of its users. It integrates features of both object-oriented and functional programming and runs on the Java platform (Java Virtual Machine).

\subsection{Language Background}
The name Scala is intended as portmanteau of scalable and language\cite{1}. The design of Scala started in 2001, at the �cole Polytechnique F�d�rale de Lausanne (EPFL) by a professor Martin Odersky. Odersky had previously worked on Java generics and \verb!javac!, Sun's Java compiler. He wanted to design a language that would combine functional programming and object-oriented programming without the restrictions imposed by Java. In his research of join calculus, he realized using its foundation would be a great way to accomplish this task. Thus, a language named Funnel was born.

Funnel was a programming language of a beautifully simple design, containing very few primitive language features. It combined the essential ideas of functional programming and Petri nets (a mathematical modeling language for the description of distributed systems)\cite{2}. This combination of ideas resulted in a very simple and expressive programming language.

However, it turned out Funnel was not a very practical language and ended up being unpleasant to use. It was designed to be minimalistic, which appealed greatly to the designers but at the same time turned away many users. Beginning users didn't understand how to do the necessary encodings, and expert users of the language got tired of having to do them over and over again. The language also lacked a set of standard libraries, making its use even less appealing. After realizing the shortcomings of Funnel, Odersky set forth to create a new language called Scala\cite{3}.

Scala's design began in 2001, and wasn't released until late 2003. It was specifically designed to be both functional and object-oriented. Data types and how objects behave are described by classes and traits. These classes can also be extended by subclassing and can be composed using mixin composition. At the same time, every function is a value that can be nested and operate on data using pattern matching. This provides support for anonymous functions and higher-order functions\cite{3}. Scala runs on the Java platform, making it compatible with existing Java programs. It even has the same compilation model as Java (and other languages depending on the version), enabling the user to call Java libraries. Those with a background in Java will find Scala's characteristics very familiar. Scala code can even be decompiled into readable Java code, with some exceptions. In the eyes of the Java Virtual Machine, Java and Scala code are the nearly the same; Scala has just one additional runtime library\cite{4}. This enables Java and other programmers to be more productive and to reducing their overall code size.

Present day, many existing companies who use Java for their applications are turning to Scala to ``boost their development productivity, applications scalability and overall reliability''\cite{1}.
Twitter, the social networking service, moved their core message queue from Ruby to Scala. The social networks popularity required a way to reliably scale their operation to meet fast growing rate of Tweets. 
With the addition of the Scala code, Twitter was able to survive large bursts of traffic more easily, allowing incoming items to be processed and delivered to the user remarkably fast, removing a lot of the application's limitations. 

\subsection{Refactoring in Scala}

Refactoring is one of the most important techniques of agile development and has been widely accepted all over the world. Scala is subject to the usual set of refactoring techniques available to most object-oriented architectures. Outlined below are some popular techniques with their principles demonstrated in Scala code.

\subsubsection{Extract Method}

One of the most commonly used refactoring techniques available in Scala is Extract Method. This technique lets you extract one or many expressions into a new private method. The refactoring takes care of passing all necessary parameters to the method and returns all values that are needed. Extract Method is a convenient technique because it makes code much more readable and easier to maintain. It also operates on a local scope, thereby avoiding dependency issues in some environments.

In order to demonstrate this technique, consider the following code:

\begin{verbatim}
val sb = new StringBuilder 
val name = new BufferedReader(new InputStreamReader(System.in)).readLine 
sb.append("Your name is: ")
sb.append(name)
\end{verbatim}

This code is pretty simple and straightforward, however it can be extracted in order to increase the code's overall reusability. We can select a portion of the code and create a private method from it:

\begin{verbatim}
def askName(out: String => Any) { 
	val name = new BufferedReader(new InputStreamReader(System.in)).readLine 
	out("Your name is: ") 
	out(name)
}
val sb = new StringBuilder askName(sb.append)
\end{verbatim}

This new method takes a function of type \verb!String -> Any! called `out' as it's parameter. Scala gives us the unique opportunity to pass an object's method as an argument instead of a function, as long as the signatures match. This can be seen with the last line of code involving the StringBuilder. This extracted method can now be used with any other function or method that satisfies the signature, greatly increasing its reusability.

\subsection{Extract Local}

Extract Local is a technique is accomplished by creating a new local variable from an expression; replacing the original expression with a reference to the new variable. This allows the user to simplify long expressions making the code much more readable. Also, by extracting new local variables we can debug a program much faster by printing out the immediate results of an expression. Consider this example:

\begin{verbatim}
object ExtractLocal {
  def main(args: Array[String]) {
    if(System.getProperties.get("os.name") == "Linux") {
      println("We're on Linux!")
    } else {
      println("We're not on Linux!")
    }
  }
}
\end{verbatim}

While this is not a badly written block of code, it could be improved. The Scala syntax and method names give us a good idea of what the intension of the expression is, but by extracting it and creating a new value we can make it much more readable:

\begin{verbatim}
object ExtractLocal {
  def main(args: Array[String]) {
    val isLinux = System.getProperties.get("os.name") == "Linux"
    if(isLinux) {
      println("We're on Linux!")
    } else {
      println("We're not on Linux!")
    }
  }
}
\end{verbatim}

\subsection{Inline Local}

Inline Local is a technique that enables the programmer to remove unneeded local values. By removing the value and replacing the references to the value with the value's content, we can make the code run much more efficiently. Consider this example:

\begin{verbatim}
object InlineLocal {
  def main(args: Array[String]) {
    val x = "This is quite unnecessary."
    println(x) 
  }
}
\end{verbatim}

By using the technique, we get:

\begin{verbatim}
object InlineLocal {
  def main(args: Array[String]) {
    println("This is quite unnecessary.") 
  }
}
\end{verbatim}

Local values can be included, but vars are not supported. This technique may be obvious and trivial to most experienced programmers, but it is still very important to use in order to keep the code readable and running efficiently.

\subsection{Conclusion}

Scala gives developers a tool to scale their applications based on its user base, ensuring its reliability and efficiency. Keeping code organized, efficient, and readable is required in order to give the users this assurance. Being a close relative of Java, most of the techniques used in refactoring are identical to those common to most object-oriented languages. Scala has a bright future ahead, and is seeing more publicity on a daily basis.

\begin{thebibliography}{9}
\bibitem{1} ``Scala Offical FAQ.'' Scala Offical FAQ. Web. 1 Nov. 2010. \verb!<http://scala-lang.org/faq/1>!.
\bibitem{2} Odersky, Martin. ``Functional Nets.'' Functional Nets. Web. 1 Nov. 2010. \verb!<http://lamp.epfl.ch/funnel/esop2000.html>!.
\bibitem{3} Odersky, Martin. ``A Brief History of Scala.'' A Brief History of Scala. Web. 2 Nov. 2010. \verb!<http://www.artima.com/weblogs/viewpost.jsp?thread=163733>!.
\bibitem{4} Pollak, David. ``For all you know, it's just another Java library.'' For all you know, it's just another Java library. Web. 2 Nov. 2010. \verb!<http://blog.lostlake.org/index.php?/archives/73-For-all-you-know,-its-just-another-Java-library.html>!.
\end{thebibliography}

\end{document}


\section{Conclusion}

Modern functional languages span a wide number of applications and implementation concepts, ranging from the mostly object-oriented to the highly mathematical. As such, the set of refactoring techniques in existence for functional languages varies as well. However, there are a few patterns that can be drawn from the languages discussed above.

Most evident is the concept that languages which approximate object-oriented behavior can be refactored with object-oriented techniques. Languages which provide classes (or an equivalent such as packages) can obviously have methods moved between classes, and so forth. In a similar vein, trivial object-oriented refactoring techniques (e.g. ``extract method'', ``rename variable'') remain in play in functional languages due to the continued existence of methods and variables.

Another prevailing trend is the similarity of some functional refactoring techniques to known mathematical practices. For example, consider the Clojure technique of composing methods: function composition is a common practice in the mathematical world, and currying -- an analogous concept in generic programming -- is taught regularly in computer science courses, primarily as a mathematical exercise. Techniques related to type inference (in languages with support for it) also fall roughly into this category.

Overall, the field of refactoring in the context of functional languages is fairly well developed. While not as ingrained into developers' consciousnesses as the object-oriented equivalent, refactoring techniques in functional languages are widely available, and can serve to improve the quality of code in almost any language.

\bibliography{Bibliography}
\bibliographystyle{plain}

\end{document}